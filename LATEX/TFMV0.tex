%%%%%%%%%%%%%%%%%%%%%%%%%%%%%%%%%%%%%%%%%%%%%%%%%%%%%%%%%%%%%%%%%%%%%%%%%%%%%%%
%                       Carga DE LA CLASE DE DOCUMENTO                        %
%                                                                             %
% Las opciones admisibles son:                                                %
%      12pt / 11pt          (cuerpo de los tipos de letra, no utilizar 10pt)  %
%                                                                             %
% catalan/spanish/english     (lengua principal del trabajo)                  %
%                                                                             % 
% french/italian/german...    (si necesitas utilizar otra lengua)             %
%                                                                             %
% listoffigures               (El documento incluye un Indice de figuras)     %
% listoftables                (El documento incluye un Indice de tablas)      %
% listofquadres               (El documento incluye un Indice de cuadros)     %
% listofalgorithms            (El documento incluye un Indice de algortimos   %
%                                                                             %
%%%%%%%%%%%%%%%%%%%%%%%%%%%%%%%%%%%%%%%%%%%%%%%%%%%%%%%%%%%%%%%%%%%%%%%%%%%%%%%

\documentclass[11pt,english,listoffigures,listoftables]{tfgetsinf}

\setcounter{tocdepth}{3}
\setcounter{secnumdepth}{3}
\usepackage[utf8]{inputenc} 
\usepackage{pifont}
\usepackage{multirow}
\usepackage{tabularx}
\usepackage{hhline}
\usepackage[section]{placeins}
\usepackage{comment}
\newcommand{\cmark}{\ding{51}}%
\newcommand{\xmark}{\ding{55}}%
\newcolumntype{Y}{>{\centering\arraybackslash}X}
%%%%%%%%%%%%%%%%%%%%%%%%%%%%%%%%%%%%%%%%%%%%%%%%%%%%%%%%%%%%%%%%%%%%%%%%%%%%%%%
%                        ALTRES PAQUETS I DEFINICIONS                         %
%                                                                             %
% Carregueu aci els paquets que necessiteu i declareu les comandes i entorns  %
%                                          (aquesta seccio pot ser buida)     %
%%%%%%%%%%%%%%%%%%%%%%%%%%%%%%%%%%%%%%%%%%%%%%%%%%%%%%%%%%%%%%%%%%%%%%%%%%%%%%%



%%%%%%%%%%%%%%%%%%%%%%%%%%%%%%%%%%%%%%%%%%%%%%%%%%%%%%%%%%%%%%%%%%%%%%%%%%%%%%%
%                        DADES DEL TREBALL                                    %
%                                                                             %
% titol, alumne, tutor i curs academic                                        %
%%%%%%%%%%%%%%%%%%%%%%%%%%%%%%%%%%%%%%%%%%%%%%%%%%%%%%%%%%%%%%%%%%%%%%%%%%%%%%%

\title{Exploring Instagram Messages on the Right to Abortion: User Profiling and Natural Language Processing Tasks}
\author{Luis López Cuerva}
\tutor{María José Castro Bleda}
\cotutors{Hurtado Oliver, Lluis Felip}
\exttutors{María Iranzo Cabrera}
\curs{2020\-2021}

%%%%%%%%%%%%%%%%%%%%%%%%%%%%%%%%%%%%%%%%%%%%%%%%%%%%%%%%%%%%%%%%%%%%%%%%%%%%%%%
%                     PARAULES CLAU/PALABRAS CLAVE/KEY WORDS                  %
%                                                                             %
% Independentment de la llengua del treball, s'hi han d'incloure              %
% les paraules clau i el resum en els tres idiomes                            %
%%%%%%%%%%%%%%%%%%%%%%%%%%%%%%%%%%%%%%%%%%%%%%%%%%%%%%%%%%%%%%%%%%%%%%%%%%%%%%%

\keywords{Processament del Llenguatge Natural, Intel·ligència Artificial, Aprenentatge Automàtic, Aprenentatge profund, xarxes socials, Instagram, entitat de nom, perfil d'usuari}  % Paraules clau 
{Procesamiento del Lenguaje Natural, Inteligencia Artificial, Aprendizaje Automático, Aprendizaje profundo, redes sociales, Instagram, entidad de nombre, perfil de usuario}  % Palabras clave
{Natural Language Processing, Artificial Intelligence, Machine Learning, Deep learning, social media, Instagram, name entity, user profiling}  % Key words           

%%%%%%%%%%%%%%%%%%%%%%%%%%%%%%%%%%%%%%%%%%%%%%%%%%%%%%%%%%%%%%%%%%%%%%%%%%%%%%%
%                              INICI DEL DOCUMENT                             %
%%%%%%%%%%%%%%%%%%%%%%%%%%%%%%%%%%%%%%%%%%%%%%%%%%%%%%%%%%%%%%%%%%%%%%%%%%%%%%%

\begin{document}

%%%%%%%%%%%%%%%%%%%%%%%%%%%%%%%%%%%%%%%%%%%%%%%%%%%%%%%%%%%%%%%%%%%%%%%%%%%%%%%
%              RESUMS DEL TFG EN VALENCIA, CASTELLA I ANGLES                  %
%%%%%%%%%%%%%%%%%%%%%%%%%%%%%%%%%%%%%%%%%%%%%%%%%%%%%%%%%%%%%%%%%%%%%%%%%%%%%%%

\begin{abstract}


\end{abstract}
\begin{abstract}[spanish]
   


\end{abstract}
\begin{abstract}[english]
   
\end{abstract}

%%%%%%%%%%%%%%%%%%%%%%%%%%%%%%%%%%%%%%%%%%%%%%%%%%%%%%%%%%%%%%%%%%%%%%%%%%%%%%%
%                              CONTINGUT DEL TREBALL                          %
%%%%%%%%%%%%%%%%%%%%%%%%%%%%%%%%%%%%%%%%%%%%%%%%%%%%%%%%%%%%%%%%%%%%%%%%%%%%%%%

\mainmatter
%%%%%%%%%%%%%%%%%%%%%%%%%%%%%%%%%%%%%%%%%%%%%%%%%%%%%%%%%%%%%%%%%%%%%%%%%%%%%%%
%                                  INTRODUCCIO                                %
%%%%%%%%%%%%%%%%%%%%%%%%%%%%%%%%%%%%%%%%%%%%%%%%%%%%%%%%%%%%%%%%%%%%%%%%%%%%%%%
\chapter{Introducción}\label{Introduccion}


\section{Motivación}
gfdfghqa

\section{Objetivos}

El trabajo aquí presentado consta de cuatro objetivos complementarios. El primero de ellos es la investigación y elección de una herramienta que permita el etiquetado manual de grandes corpus de datos, el segundo es la investigación, análisis y uso de diversas herramientas de procesamiento de lenguaje natural que permiten realizar análisis de polaridad a nivel aspectual, el tercero es etiquetar el conjunto de datos, el cuarto es realizar un análisis de la experiencia de usuario mediante la información obtenida de la experimentación realizada.

Para evaluar el grado de cumplimiento de estos objetivos se plantea la siguiente rúbrica \ref{fig:RubricaObjetivos}, la cual se utilizará al final del trabajo para ver hasta qué punto se han cumplido los objetivos.a

\begin{table}[!ht]
   \centering
   \resizebox{\textwidth}{!}{
   \begin{tabularx}{\textwidth}{|Y|Y|Y|Y|Y|}
   \hline
   \textbf{Objetivos}                     & \textbf{Objetivo no cumplido} & \textbf{Objetivo trabajado insuficientemente} & \textbf{Objetivo alcanzado suficientemente} & \textbf{Objetivo cumplido} \\ \hhline{|=|=|=|=|=|}
Investigar y elegir de plataforma de etiquetado & No se han investigado plataformas de etiquetado. & Se ha elegido una plataforma de etiquetado. & Se han investigado varias plataformas de etiquetado. & Se han investigado y comparado varias plataformas de etiquetado para finalmente utilizar una durante este trabajo.\\ \hline
Investigar, analizar y utilizar diversas plataformas de NLP  & No se ha investigado ninguna plataforma. &  Se han investigado y analizado múltiples herramientas. & Se han investigado y analizando diversas herramientas. Se han utilizado varias herramientas de NLP. & Se han investigado, analizado, usado y evaluado varias herramientas de NLP. \\ \hline
Etiquetar el conjunto de datos   & No se han etiquetado los datos. & Los datos han sido etiquetados por una sola persona.  & Los datos han sido etiquetados por varias personas. & Los datos han sido etiquetados por varias personas y se ha validado el etiquetado estadísticamente.  \\ \hline
Realizar un análisis de la experiencia de usuario   & No se ha realizado un análisis de la experiencia de usuario & Se han empleado los datos etiquetados para realizar un análisis estadístico de la UX. & Se ha realizado un análisis de la UX a partir de las categorías de aspectos. & Se ha realizado un análisis de la UX relacionando los datos obtenidos en este trabajo con otros trabajos punteros del campo. \\ \hline

   \end{tabularx}
   }
   \caption{\textbf{Rúbrica de evaluación de objetivos.}}
   \label{fig:RubricaObjetivos}
\end{table}
\newpage
\section{Estructura de la memoria}

El presente trabajo se divide en ocho capítulos, bibliografía y dos anexos. En cada uno de ellos se presenta el estado actual de la tecnología, un conjunto de herramientas, los análisis realizados, la experimentación o un análisis final de la experiencia de usuario. Concretamente los capítulos presentes son los siguientes:
\begin{enumerate}
   \item \nameref{Introduccion}: presenta este trabajo y sus objetivos.
   \item \nameref{EstadoArte}: expone el estado actual de los diversos estudios relacionados con este trabajo.
\end{enumerate}

Después de estos capítulos se listan las referencias bibliográficas que se han utilizado en el trabajo. La memoria finaliza con dos anexos, el anexo \ref{Anexo A}.


%\section{Notes bibliografiques} %%%%% Opcional


%%%%%%%%%%%%%%%%%%%%%%%%%%%%%%%%%%%%%%%%%%%%%%%%%%%%%%%%%%%%%%%%%%%%%%%%%%%%%%%
%                         Estado del arte                                     %
%%%%%%%%%%%%%%%%%%%%%%%%%%%%%%%%%%%%%%%%%%%%%%%%%%%%%%%%%%%%%%%%%%%%%%%%%%%%%%%

\chapter{Estado del arte} \label{EstadoArte} 
\cite{MVS2000}

\section{Análisis de sentimiento}


\section{Extracción de aspectos}

\section{Experiencia de usuario en el aprendizaje virtual}


%%%%%%%%%%%%%%%%%%%%%%%%%%%%%%%%%%%%%%%%%%%%%%%%%%%%%%%%%%%%%%%%%%%%%%%%%%%%%%%
%                Herramientas procesamiento del lenguaje natural              %
%%%%%%%%%%%%%%%%%%%%%%%%%%%%%%%%%%%%%%%%%%%%%%%%%%%%%%%%%%%%%%%%%%%%%%%%%%%%%%%

\section{Trabajos futuros}

%modelos propios \cite{neuralproclett}  transfer learning \cite{a transfer learning}...


%%%%%%%%%%%%%%%%%%%%%%%%%%%%%%%%%%%%%%%%%%%%%%%%%%%%%%%%%%%%%%%%%%%%%%%%%%%%%%%
%                                BIBLIOGRAFIA                                 %
%%%%%%%%%%%%%%%%%%%%%%%%%%%%%%%%%%%%%%%%%%%%%%%%%%%%%%%%%%%%%%%%%%%%%%%%%%%%%%%
\bibliographystyle{unsrt}
%\bibliographystyle{ieeetr}
\bibliography{Bibliografia}

\cleardoublepage

%%%%%%%%%%%%%%%%%%%%%%%%%%%%%%%%%%%%%%%%%%%%%%%%%%%%%%%%%%%%%%%%%%%%%%%%%%%%%%%
%                           APÈNDIXS  (Si n'hi ha!)                           %
%%%%%%%%%%%%%%%%%%%%%%%%%%%%%%%%%%%%%%%%%%%%%%%%%%%%%%%%%%%%%%%%%%%%%%%%%%%%%%%

\APPENDIX
%%%%%%%%%%%%%%%%%%%%%%%%%%%%%%%%%%%%%%%%%%%%%%%%%%%%%%%%%%%%%%%%%%%%%%%%%%%%%%%
%                         LA CONFIGURACIO DEL SISTEMA                         %
%%%%%%%%%%%%%%%%%%%%%%%%%%%%%%%%%%%%%%%%%%%%%%%%%%%%%%%%%%%%%%%%%%%%%%%%%%%%%%%
\chapter{Anexo A}\label{Anexo A}

%%%%%%%%%%%%%%%%%%%%%%%%%%%%%%%%%%%%%%%%%%%%%%%%%%%%%%%%%%%%%%%%%%%%%%%%%%%%%%%
%                              FI DEL DOCUMENT                                %
%%%%%%%%%%%%%%%%%%%%%%%%%%%%%%%%%%%%%%%%%%%%%%%%%%%%%%%%%%%%%%%%%%%%%%%%%%%%%%%

\end{document}
